\documentclass[11pt,a4]{jsarticle}
\pagestyle{empty}
\setcounter{secnumdepth}{3}
\setcounter{tocdepth}{2}
\bibliographystyle{h-physrev3}
\usepackage[T1]{fontenc}
\usepackage[active]{srcltx}
\usepackage[version=3]{mhchem}

\makeatletter
%\usepackage{tgtermes}
%\usepackage[T1]{fontenc}
\usepackage[top=10truemm,bottom=10truemm,left=20truemm,right=20truemm]{geometry}
\usepackage{fancyhdr, lastpage}
\usepackage{amsmath}
%\usepackage[lite,subscriptcorrection,slantedGreek,nofontinfo]{mtpro2}
\usepackage{bm,braket,ascmac,enumerate,multirow}
\usepackage{amssymb,wrapfig,afterpage,booktabs,url}
\usepackage{listings}
\numberwithin{equation}{section}
%
\usepackage{graphicx}
\usepackage[dvips]{color}
\usepackage{makeidx}
\usepackage{fancyvrb}
\usepackage{cprotect}
\usepackage[dvipdfmx]{hyperref}
\usepackage{pxjahyper}
%
\newcommand{\vred}{\color{red}}
\newcommand{\vblue}{\color{blue}}
\newcommand{\vgreen}{\color{green}}
\newcommand{\at}{@}
%
\renewcommand{\baselinestretch}{1.1}
\renewcommand{\figurename}{Fig.}
\renewcommand{\tablename}{Table }

\makeindex 
%\renewcommand{\contentsname}{\Large \centerline{目 次}}
%\renewcommand{\figurename}{Fig.}
%\renewcommand{\tablename}{Table }
%\renewcommand{\refname}{}

\makeatother

\newcommand{ \Sec }[1]{Sec.~\ref{sec:#1}}
\newcommand{ \Appendix }[1]{Appendix \ref{sec:#1}}

\newcommand{ \Eq   }[1]{Eq.~(\ref{#1})}
\newcommand{ \Eqs  }[2]{Eqs.~(\ref{#1}) and (\ref{#2})}
\newcommand{ \Equation }[1]{Equation (\ref{#1})}

\newcommand{ \Table }[1]{Table \ref{tab:#1}}

%\newcommand{ \Ref  }[1]{Ref.~\onlinecite{#1}}
\newcommand{ \Refs }[2]{Refs.~\onlinecite{#1} and \onlinecite{#2}}

\newcommand{ \Fig     }[1]{Fig.~\ref{fig:#1}}
\newcommand{ \Figs    }[2]{Figs.~\ref{fig:#1} and \ref{fig:#2}}
\newcommand{ \Figure  }[1]{Figure \ref{fig:#1}}
%.........................................................

%.........................................................
\newtheorem{reidai}{例題}
\newtheorem{enshu}{演習問題}
%
\begin{document}
\section*{第7回 演習問題(2021/07/14)}
%
\hrule
\vspace*{.2cm}
\enshu
波動方程式
\begin{align*}
 \dfrac{\partial^{2}}{\partial t^{2}}u\left(x,t\right) & =v^{2}\dfrac{\partial^{2}}{\partial x^{2}}u\left(x,t\right),\quad v>0,\,0\leq x\leq L
\end{align*}
を次の初期条件,境界条件の下で解け.
\begin{align*}
  & u\left(x,0\right)=\begin{cases}
x & 0\leq x<\dfrac{L}{2}\\
-x+L & \dfrac{L}{2}\leq x\leq L
\end{cases},\\
 & \dfrac{\partial}{\partial t}u\left(x,t\right)\biggr|_{t=0}=0, \\
 & \textcolor{red}{u(0,t) = u(L,t) = 0.}  
\end{align*}
ただし,波動現象として不適切な解(時刻$t\to \infty$で発散するようなもの)は除外して良い.
\vspace*{.2cm}
\hrule
\vspace*{.2cm}
%
略解をあらかじめ書いておくと,
\begin{align*}
 u\left(x,t\right) & =\sum_{n=1}^{\infty}\dfrac{4L}{\left(n\pi\right)^{2}}\sin\left(\dfrac{\pi n}{2}\right)\sin\left(\dfrac{n\pi}{L}x\right)\cos\left(\dfrac{n\pi}{L}vt\right),
\end{align*}
です.
\end{document}
