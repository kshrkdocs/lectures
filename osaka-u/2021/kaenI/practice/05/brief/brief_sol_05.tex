\documentclass[11pt,a4]{jsarticle}
\pagestyle{empty}
\setcounter{secnumdepth}{3}
\setcounter{tocdepth}{2}
\bibliographystyle{h-physrev3}
\usepackage[T1]{fontenc}
\usepackage[active]{srcltx}
\usepackage[version=3]{mhchem}

\makeatletter
%\usepackage{tgtermes}
%\usepackage[T1]{fontenc}
\usepackage[top=10truemm,bottom=10truemm,left=20truemm,right=20truemm]{geometry}
\usepackage{fancyhdr, lastpage}
\usepackage{amsmath}
%\usepackage[lite,subscriptcorrection,slantedGreek,nofontinfo]{mtpro2}
\usepackage{bm,braket,ascmac,enumerate,multirow}
\usepackage{amssymb,wrapfig,afterpage,booktabs,url}
\usepackage{listings}
\numberwithin{equation}{section}
%
\usepackage{graphicx}
\usepackage[dvips]{color}
\usepackage{makeidx}
\usepackage{fancyvrb}
\usepackage{cprotect}
\usepackage[dvipdfmx]{hyperref}
\usepackage{pxjahyper}
%
\newcommand{\vred}{\color{red}}
\newcommand{\vblue}{\color{blue}}
\newcommand{\vgreen}{\color{green}}
\newcommand{\at}{@}
%
\renewcommand{\baselinestretch}{1.1}
\renewcommand{\figurename}{Fig.}
\renewcommand{\tablename}{Table }

\makeindex 
%\renewcommand{\contentsname}{\Large \centerline{目 次}}
%\renewcommand{\figurename}{Fig.}
%\renewcommand{\tablename}{Table }
%\renewcommand{\refname}{}

\makeatother

\newcommand{ \Sec }[1]{Sec.~\ref{sec:#1}}
\newcommand{ \Appendix }[1]{Appendix \ref{sec:#1}}

\newcommand{ \Eq   }[1]{Eq.~(\ref{#1})}
\newcommand{ \Eqs  }[2]{Eqs.~(\ref{#1}) and (\ref{#2})}
\newcommand{ \Equation }[1]{Equation (\ref{#1})}

\newcommand{ \Table }[1]{Table \ref{tab:#1}}

%\newcommand{ \Ref  }[1]{Ref.~\onlinecite{#1}}
\newcommand{ \Refs }[2]{Refs.~\onlinecite{#1} and \onlinecite{#2}}

\newcommand{ \Fig     }[1]{Fig.~\ref{fig:#1}}
\newcommand{ \Figs    }[2]{Figs.~\ref{fig:#1} and \ref{fig:#2}}
\newcommand{ \Figure  }[1]{Figure \ref{fig:#1}}
%.........................................................

%.........................................................
\newtheorem{reidai}{例題}
\newtheorem{enshu}{演習問題}
%
\begin{document}
%
\section*{第5回 演習問題(2021/06/16) 略解とヒント}
\begin{flushright}
 最終更新日 : 2021/07/04 
\end{flushright}
\hrule
%\vspace*{.2cm}
\enshu
%
テキストの「ラプラス変換の性質」と「ラプラス逆変換の性質」について復習し,
自分なりにまとめてみよ.式の羅列のみは採点しない.
また,テキストでは導出を省略した
\begin{align*}
 &\mathcal{L}\left[f^{(n)}(t)\right] = s^n F(s) - s^{n-1}f(0) - s^{n-2}f^{\prime}(0) - \cdots -sf^{(n-2)}(0) - f^{(n-1)}(0), \\
 &\mathcal{L}^{-1}\left[F^{(n)}(s)\right] = (-t)^{n}f(t),
\end{align*}
を導出してみよ.
\vspace*{.2cm}
\hrule
\vspace*{.2cm}
(省略)
%
\newpage
%
\hrule
\enshu
%
次の関数$f(t)$のラプラス変換$F(s)$を求めよ.導出過程も含めて書くこと.
(ラプラス変換表より...で,解答を終わらせないこと.もちろん,答え合わせには使っても良い)
\begin{enumerate}[(1)]
  %\item $f(t) = \dfrac{1}{\sqrt{\pi t}}$
  \item $f(t) = \sinh(at)$
  \item $f(t) = \cos^{2}(at)$
  %\item $f(t) = \dfrac{1}{t}\left(e^{-at}-e^{-bt}\right)$
  %\item $f(t) = e^{-at}\cos \omega_0 t$
  \item $f(t) = te^{-at}\cos\omega_0 t$
\end{enumerate}
%
\hrule
\vspace*{.2cm}
\begin{enumerate}[(1)]
  \item 
    \begin{align*}
      F(s) = \dfrac{a}{s^2 - a^2}
    \end{align*}
  \item
    \begin{align*}
      F(s) = \dfrac{s^2 + 2a^2}{s^3 + 4a^2 s} 
    \end{align*}
  \item
    \begin{align*}
      F(s) = \dfrac{\left(s+a\right)^{2} - \omega_{0}^{2}}{\left((s+a)^2 + \omega_{0}^{2}\right)^{2}} 
    \end{align*}
\end{enumerate}

\newpage
%
\hrule
\enshu
%
次の関数$F(s)$のラプラス逆変換$f(t)$を求めよ.テキストのラプラス変換表は用いて良い.
\begin{enumerate}[(1)]
  \item $F(s) = \dfrac{1}{s-2}$
  \item $F(s) = \dfrac{1}{s^2+2s}$
  %\item $F(s) = \dfrac{s}{s^2-3}$ 
\end{enumerate}
\hrule
\vspace*{.2cm}
%
\begin{enumerate}[(1)]
  \item 
    \begin{align*}
      f(t) = e^{2t} 
    \end{align*}
  \item
    \begin{align*}
      f(t) = \dfrac{1}{2}\left(1-e^{-2t}\right) 
    \end{align*}
\end{enumerate}
%
\newpage
%
\hrule
\enshu
%
2階の線形常微分方程式を自分で作ってみて,それをラプラス変換で解いてみよ.
ただし,テキストと同じ式にはしないこと.
%
\vspace*{.2cm}
\hrule
\vspace*{.2cm}
(省略)
\end{document}
