\section*{第5回 演習問題(2021/06/16)}
%
\enshu
%
テキストの「ラプラス変換の性質」と「ラプラス逆変換の性質」について復習し,
自分なりにまとめてみよ.式の羅列のみは採点しない.
また,テキストでは導出を省略した
\begin{align}
 &\mathcal{L}\left[f^{(n)}(t)\right] = s^n F(s) - s^{n-1}f(0) - s^{n-2}f^{\prime}(0) - \cdots -sf^{(n-2)}(0) - f^{(n-1)}(0), \\
 &\mathcal{L}^{-1}\left[F^{(n)}(s)\right] = (-t)^{n}f(t),
\end{align}
を導出してみよ.

%
\enshu
%
次の関数$f(t)$のラプラス変換$F(s)$を求めよ.導出過程も含めて書くこと.
(ラプラス変換表より...で,解答を終わらせないこと.もちろん,答え合わせには使っても良い)
\begin{enumerate}[(1)]
  %\item $f(t) = \dfrac{1}{\sqrt{\pi t}}$
  \item $f(t) = \sinh(at)$
  \item $f(t) = \cos^{2}(at)$
  %\item $f(t) = \dfrac{1}{t}\left(e^{-at}-e^{-bt}\right)$
  %\item $f(t) = e^{-at}\cos \omega_0 t$
  \item $f(t) = te^{-at}\cos\omega_0 t$
\end{enumerate}
%
\enshu
%
次の関数$F(s)$のラプラス逆変換$f(t)$を求めよ.テキストのラプラス変換表は用いて良い.
\begin{enumerate}[(1)]
  \item $F(s) = \dfrac{1}{s-2}$
  \item $F(s) = \dfrac{1}{s^2+2s}$
  %\item $F(s) = \dfrac{s}{s^2-3}$ 
\end{enumerate}
%
\enshu
%
2階の線形常微分方程式を自分で作ってみて,それをラプラス変換で解いてみよ.
ただし,テキストと同じ式にはしないこと.