\section*{第2回 演習問題(2021/04/22)}
%
\enshu
化学反応
\begin{align*}
  \ce{A + B <=>[k][k^{\prime}] C + D}, 
\end{align*}
を考える.ただし,$k,~k^{\prime}$はそれぞれ2次,1次反応の速度定数であり,
A, B, C, Dの初期濃度を$a,~2a,~0,~0$とする.
\begin{enumerate}[(1)]
  \item Cの時刻$t$での濃度を$x(t)$と表すとき,
	$x(t)$に関する微分方程式を立てよ.
  \item $k^\prime = 2k$のとき,(1)で導出した微分方程式を解け. 
\end{enumerate}
%
\enshu
以下の問いに答えよ.
\begin{enumerate}[(1)]
  \item 非線形微分方程式
	\begin{align*}
	   \dfrac{dy}{dx} = p(x)y + q(x)y^{\alpha},\quad (\alpha \neq 0, 1),
	\end{align*}
	は$u=y^{1-\alpha}$と変数変換することにより,線形方程式に帰着することを示せ.
  \item 微分方程式
	\begin{align*}
	  3x\dfrac{dy}{dx} + y = xy^5, 
	\end{align*}
	を解け.
\end{enumerate}
%
\enshu
以下の問いに答えよ.
\begin{enumerate}[(1)]
  \item 微分方程式
	\begin{align*}
	  \dfrac{dy}{dx} = a(x)y^2 + b(x)y + c(x), 
	\end{align*}
	の特殊解が既に一つ求まっており,それを$y_0$とすると,
	変数変換$u = y - y_0$により,与えれた微分方程式は
	\begin{align*}
	  \dfrac{du}{dx} = p(x)u + q(x)u^{2}, 
	\end{align*}
	の形に帰着することを示せ.
\end{enumerate}
%
\enshu
化学反応
\begin{align*}
 \ce{A  <=>[k_1][k_{-1}] B <=>[k_2][k_{-2}] C}, 
\end{align*}
を考える.A, B, Cの時刻$t$での濃度をそれぞれ
$x\left(t\right),~y\left(t\right),~z\left(t\right)$,
初期濃度を$x_0,~y_0,~z_0$とする.
\begin{enumerate}[(1)]
  \item $x(t),~y(t),~z(t)$に対する反応速度式(微分方程式)を立てよ.
  \item (1)で立てた微分方程式のうち,$x(t)$に関する方程式を解け.
	必要に応じて追加で定数を導入しても良い.ただし,その定義は書いておくこと.
\end{enumerate}
%
\newpage
%
\enshu
微分方程式
\begin{align*}
 \dfrac{d^2x}{dt^2} + 2\gamma \dfrac{dx}{dt} + \omega_{0}^{2}x = f_0 \cos \omega t, 
\end{align*}
を考える\footnote{この方程式の斉次形が何者かは第1回の演習問題4を参照すること.粘性流体の中でバネに繋がれたおもりに,外力$f_0 \cos \omega t$が加えられた状態に対応するのが今回の微分方程式である.}.ただし,$0< \gamma < \omega_{0}$とする.
\begin{enumerate}[(1)]
  \item 一般解を求めよ.
  \item 初期条件を$x(0)=0, d x(t)/dt|_{t=0}=0$として,$x(t)$の概形を描いてみよ.
	また$t$が十分に大きいときの$x(t)$を求めよ.
\end{enumerate}
%
\enshu (べき級数解法)
%

講義では省略したべき級数解法について,雰囲気だけでも掴んでおこう.
例えば,次の微分方程式を考える.
\begin{align*}
  \dfrac{dy}{dx} - y = 0. 
\end{align*}
既に知っているように,一般解は$y=Ce^{x}$である.
ここでは一般解をべき級数解法を用いて導いてみる.

どのような$y$が方程式の解になっているか全く見当がつかないとする(実際にはついているとしても).
そこで,どんな形であっても対応できるように,$y$をべき級数の形で表しておくことにする.
\begin{align*}
  y = \sum_{n=0}^{\infty} a_{n}x^{n}. 
\end{align*}
これを与えられた微分方程式に代入すると,
\begin{align*}
  \dfrac{dy}{dx} - y &= \sum_{n=0}^{\infty}\left(n+1\right)a_{n+1}x^n 
                        - \sum_{n=0}^{\infty}a_n x^{n} \notag \\
                     &= \sum_{n=0}^{\infty}\underline{\left\{(n+1)a_{n+1}-a_n\right\}}x^{n} \notag \\
                     &= 0, 
\end{align*}
となる.これが$x$の値によらず成り立つためには,
全ての$x^{n}$についての係数(下線部)がゼロ
でなければならない.つまり,
\begin{align*}
  &(n+1)a_{n+1} - a_n = 0, \notag \\
  &\rightarrow a_{n+1} = \dfrac{1}{n+1}a_{n},
\end{align*}
という漸化式が得られ,一般項を求めると,
\begin{align*}
 a_n = \dfrac{1}{n!}a_0,
\end{align*}
である.従って,微分方程式の解が,
\begin{align*}
  y = a_0 \sum_{n=0}^{\infty}\dfrac{1}{n!}x^n = a_{0}e^{x} 
\end{align*}
となることを示せた.

上記の例を参考にして,微分方程式
\begin{align*}
  \dfrac{d^2y}{dx^2} + y = 0,
\end{align*}
を解いてみよう.
