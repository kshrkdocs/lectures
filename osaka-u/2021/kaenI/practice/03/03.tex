\section*{第3回 演習問題(2021/05/20)}
%
\enshu
%
次のの関数$f(x)$について,フーリエ級数展開を実行せよ.
\begin{enumerate}[(1)]
  \item 周期1の関数 
	\begin{align*}
	  f(x)=x \quad (-1/2\leq x < 1/2)
	\end{align*}
  \item 周期$2\pi$の関数 
	\begin{align*}
	  f(x) = 
	  \begin{cases}
	    \dfrac{x}{\pi} & -\pi \leq x < 0 \\
            1              & 0 \leq x < \pi 
	  \end{cases}
	\end{align*}
  \item 周期$2\pi$の関数
	\begin{align*}
	  f(x) = x^2\quad (-\pi \leq x < \pi)
	\end{align*}
\end{enumerate}
%
\enshu
以下の問いに答えよ.
\begin{enumerate}[(1)]
  \item 周期$2$の関数
	\begin{align*}
	  f(x) = x+1, \quad (-1\leq x < 1) 
	\end{align*}
	をフーリエ級数展開せよ.
  \item 関数
	\begin{align*}
	  f(x) = x, \quad (0\leq x < 1) 
	\end{align*}
	のフーリエ正弦級数,余弦級数を求めよ.
\end{enumerate}

%
\enshu
%
以下の問いに答えよ.
\begin{enumerate}[(1)]
\item 
次式で定義される
\begin{align*}
 \delta (u) = \dfrac{1}{2\pi}\int_{-\infty}^{\infty}dt\,e^{-itu}
\end{align*}
と任意の$f(u)$に対して次式が成り立つことを示せ.
\begin{align*}
 \int_{-\infty}^{\infty}du\,\delta(u)f(u)=f(0). 
\end{align*}
\item (1)で定義された$\delta(u)$はディラックのデルタ関数と呼ばれるものであり,これから皆さんと長い付き合いになる.
      デルタ関数について調べて,その性質を(分かる範囲で良いので)まとめてみよ.
\end{enumerate}