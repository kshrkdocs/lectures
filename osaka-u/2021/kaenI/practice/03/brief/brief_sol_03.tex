\documentclass[11pt,a4]{jsarticle}
\pagestyle{empty}
\setcounter{secnumdepth}{3}
\setcounter{tocdepth}{2}
\bibliographystyle{h-physrev3}
\usepackage[T1]{fontenc}
\usepackage[active]{srcltx}
\usepackage[version=3]{mhchem}

\makeatletter
%\usepackage{tgtermes}
%\usepackage[T1]{fontenc}
\usepackage[top=10truemm,bottom=10truemm,left=20truemm,right=20truemm]{geometry}
\usepackage{fancyhdr, lastpage}
\usepackage{amsmath}
%\usepackage[lite,subscriptcorrection,slantedGreek,nofontinfo]{mtpro2}
\usepackage{bm,braket,ascmac,enumerate,multirow}
\usepackage{amssymb,wrapfig,afterpage,booktabs,url}
\usepackage{listings}
\numberwithin{equation}{section}
%
\usepackage{graphicx}
\usepackage[dvips]{color}
\usepackage{makeidx}
\usepackage{fancyvrb}
\usepackage{cprotect}
\usepackage[dvipdfmx]{hyperref}
\usepackage{pxjahyper}
%
\newcommand{\vred}{\color{red}}
\newcommand{\vblue}{\color{blue}}
\newcommand{\vgreen}{\color{green}}
\newcommand{\at}{@}
%
\renewcommand{\baselinestretch}{1.1}
\renewcommand{\figurename}{Fig.}
\renewcommand{\tablename}{Table }

\makeindex 
%\renewcommand{\contentsname}{\Large \centerline{目 次}}
%\renewcommand{\figurename}{Fig.}
%\renewcommand{\tablename}{Table }
%\renewcommand{\refname}{}

\makeatother

\newcommand{ \Sec }[1]{Sec.~\ref{sec:#1}}
\newcommand{ \Appendix }[1]{Appendix \ref{sec:#1}}

\newcommand{ \Eq   }[1]{Eq.~(\ref{#1})}
\newcommand{ \Eqs  }[2]{Eqs.~(\ref{#1}) and (\ref{#2})}
\newcommand{ \Equation }[1]{Equation (\ref{#1})}

\newcommand{ \Table }[1]{Table \ref{tab:#1}}

%\newcommand{ \Ref  }[1]{Ref.~\onlinecite{#1}}
\newcommand{ \Refs }[2]{Refs.~\onlinecite{#1} and \onlinecite{#2}}

\newcommand{ \Fig     }[1]{Fig.~\ref{fig:#1}}
\newcommand{ \Figs    }[2]{Figs.~\ref{fig:#1} and \ref{fig:#2}}
\newcommand{ \Figure  }[1]{Figure \ref{fig:#1}}
%.........................................................

%.........................................................
\newtheorem{reidai}{例題}
\newtheorem{enshu}{演習問題}
%
\begin{document}
\section*{第3回 演習問題(2021/05/20) 略解とヒント}
%
\begin{flushright}
 最終更新日 : 2021/07/04 
\end{flushright}
%
この回の演習の問題はどれも,皆さんよく出来ていました.
計算間違いはあるものの,基本的な方針から誤っている人はほとんどいませんでしたので,
この回は略解のみを示します.(もし,略解に誤りがある場合は適宜修正します)
%
\vspace*{.2cm}
%
\hrule
\vspace*{.2cm}
\enshu
%
次の関数$f(x)$について,フーリエ級数展開を実行せよ.
\begin{enumerate}[(1)]
  \item 周期1の関数 
	\begin{align*}
	  f(x)=x \quad (-1/2\leq x < 1/2)
	\end{align*}
  \item 周期$2\pi$の関数 
	\begin{align*}
	  f(x) = 
	  \begin{cases}
	    \dfrac{x}{\pi} & -\pi \leq x < 0 \\
            1              & 0 \leq x < \pi 
	  \end{cases}
	\end{align*}
  \item 周期$2\pi$の関数
	\begin{align*}
	  f(x) = x^2\quad (-\pi \leq x < \pi)
	\end{align*}
\end{enumerate}
%
\hrule
\vspace*{.2cm}
%
\begin{enumerate}[(1)]
  \item
    \begin{align*}
      f\left(x\right) & =\sum_{n=1}^{\infty}\dfrac{\left(-1\right)^{n+1}}{\pi n}\sin\left(2\pi nx\right)
    \end{align*}
  \item
    \begin{align*}
      f\left(x\right) & =\dfrac{1}{4}+\sum_{n=1}^{\infty}\dfrac{1}{n^{2}\pi^{2}}\left(1+\left(-1\right)^{n+1}\right)\cos\left(nx\right)+\sum_{n=1}^{\infty}\dfrac{1}{n\pi}\left(2\left(-1\right)^{n+1}+1\right)\sin\left(nx\right)
    \end{align*}
  \item 
    \begin{align*}
      f\left(x\right) & =\dfrac{\pi^{2}}{3}+4\sum_{n=1}^{\infty}\dfrac{\left(-1\right)^{n}}{n^{2}}\cos nx
    \end{align*} 
\end{enumerate}
%
\newpage
%
\hrule
\enshu
以下の問いに答えよ.
\begin{enumerate}[(1)]
  \item 周期$2$の関数
	\begin{align*}
	  f(x) = x+1, \quad (-1\leq x < 1) 
	\end{align*}
	をフーリエ級数展開せよ.
  \item 関数
	\begin{align*}
	  f(x) = x, \quad (0\leq x < 1) 
	\end{align*}
	のフーリエ正弦級数,余弦級数を求めよ.
\end{enumerate}
\hrule
%\vspace*{.2cm}
%
\begin{enumerate}[(1)]
  \item 
    \begin{align*}
      f(x) = 1 + \sum_{n=1}^{\infty}\dfrac{2(-1)^{n+1} }{\pi n} \sin (\pi nx) 
    \end{align*}
  \item
    正弦級数の場合は,
      \begin{align*}
        f\left(x\right) & =\sum_{n=1}^{\infty}\dfrac{2n\left(-1\right)^{n+1}}{\pi}\sin\left(n\pi x\right)   
      \end{align*}
    余弦級数の場合は,
      \begin{align*}
	f\left(x\right) & =\dfrac{1}{2}-\dfrac{4}{\pi^{2}}\sum_{n=1}^{\infty}\dfrac{1}{\left(2n-1\right)^{2}}\cos\left(\left(2n-1\right)\pi x\right)
      \end{align*}
\end{enumerate}
%
\newpage
%
\hrule
\enshu
%
以下の問いに答えよ.
\begin{enumerate}[(1)]
\item 
次式で定義される
\begin{align*}
 \delta (u) = \dfrac{1}{2\pi}\int_{-\infty}^{\infty}dt\,e^{-itu}
\end{align*}
と任意の$f(u)$に対して次式が成り立つことを示せ.
\begin{align*}
 \int_{-\infty}^{\infty}du\,\delta(u)f(u)=f(0). 
\end{align*}
\item (1)で定義された$\delta(u)$はディラックのデルタ関数と呼ばれるものであり,これから皆さんと長い付き合いになる.
      デルタ関数について調べて,その性質を(分かる範囲で良いので)まとめてみよ.
\end{enumerate}
%
\hrule
%
\vspace*{.2cm}
テキストにまとめがあるので,省略します.
\end{document}
