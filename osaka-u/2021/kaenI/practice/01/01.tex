\section*{第1回 演習問題(2021/04/15)}
%
\enshu
放射性物質の量は時間がたつと減少していくことが知られている.
時刻$t$での放射性粒子(原子)の数を$N(t)$と表すと,
$N(t)$の時間変化率はその時刻での粒子数$N(t)$に比例することが実験的に確かめられている.
\begin{enumerate}[(1)]
  \item 比例定数を$\lambda$ ($\lambda>0$,崩壊定数と呼ばれる) として,$N(t)$に関する微分方程式を立てよ.
  \item (1)で導出した微分方程式を解け.ただし,時刻$t=0$での粒子数を$N_0$とする.
  \item 粒子数が半分になるのに要する時間$\tau$を求めよ.
\end{enumerate}

\enshu
微分方程式
\begin{align*}
  \dfrac{dy}{dx} + 2y = 2\sin x, 
\end{align*}
の一般解を求めよ.また,初期条件が$y(0)=1$のときの解を求めよ.

\enshu
化学反応
\begin{align*}
 \ce{A ->[k_1] B ->[k_2] C}  
\end{align*}
を考える.ただし,$k_1,~k_2$は各過程における速度定数である.
A, B, Cの時刻$t$での濃度をそれぞれ$\ce{[A]}_t,~\ce{[B]}_t,~\ce{[C]}_t$
とすると,反応速度式(微分方程式)は
\begin{align*}
  &\dfrac{d[\mathrm{A}]_t}{dt} = -k_1 [\mathrm{A}]_t, \\
  &\dfrac{d[\mathrm{B}]_t}{dt} =  k_1 [\mathrm{A}]_t - k_2 [\mathrm{B}]_t, \\
  &\dfrac{d[\mathrm{C}]_t}{dt} =  k_2 [\mathrm{B}], 
\end{align*}
で表される.
初期濃度を$[\mathrm{A}]_0,~[\mathrm{B}]_0,~[\mathrm{C}]_0$として上記の微分方程式を解いて,
A, B, Cの濃度の時間変化を求めよ.

\enshu
バネにつながれた質量$m$のおもりの運動$x(t)$を考える.
おもりの平衡位置を$x=0$,バネ定数を$k$とすると,このおもりの運動は次の微分方程式に
従うとする.
\begin{align*}
  m\dfrac{d^2x}{dt^2} = -k x. 
\end{align*}
いま,このおもりとバネを丸ごと粘性のある流体の中に入れる.
すると,おもりが動く方向とは逆向きに力(粘性力)が働く.
おもりの動く速さがあまり大きくないときは,その力はおもりの速度に比例する.ここでは比例定数を$\Gamma ~(>0)$と表すことにする.
\begin{enumerate}[(1)]
  \item 流体の中でのおもりの運動方程式を立てよ.
  \item $\omega=\sqrt{k/m}$, $\gamma = \Gamma/(2m)$とおいて,次のそれぞれの場合での(1)で立てた微分方程式の解を求めよ.
	ただし,$x(0)=1$, $dx/dt|_{t=0} = 0$とする.\\
	(i) $\gamma < \omega$, (ii) $\gamma = \omega$, (iii) $\gamma > \omega$
  \item (i), (ii), (iii)における解$x(t)$を時間$t$に対してプロットしてみて,その形について考察せよ.
	プロットは手書きで大まかに書いても良いし,ExcelやGnuplotを使っても良い.
\end{enumerate}
