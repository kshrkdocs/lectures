\section*{第1回 演習問題(2021/04/15) 解答例}
%
\hrule
\vspace*{.2cm}
\enshu
放射性物質の量は時間がたつと減少していくことが知られている.
時刻$t$での放射性粒子(原子)の数を$N(t)$と表すと,
$N(t)$の時間変化率はその時刻での粒子数$N(t)$に比例することが実験的に確かめられている.
\begin{enumerate}[(1)]
  \item 比例定数を$\lambda$ ($\lambda>0$,崩壊定数と呼ばれる) として,$N(t)$に関する微分方程式を立てよ.
  \item (1)で導出した微分方程式を解け.ただし,時刻$t=0$での粒子数を$N_0$とする.
  \item 粒子数が半分になるのに要する時間$\tau$を求めよ.
\end{enumerate}
\hrule
\vspace*{.2cm}

\setcounter{section}{1}
\noindent
(1)
\begin{align}
 \dfrac{dN}{dt} = -\lambda N 
\end{align}
(2)
\begin{align}
  N(t) = N_{0}e^{-\lambda t}
\end{align}
(3) $\tau$が満たすべき方程式は
\begin{align}
  \dfrac{N_0}{2} = N_0 e^{-\lambda \tau} 
\end{align}
である.これを$\tau$について解くと,
\begin{align}
  \tau = \dfrac{1}{\lambda} \log 2 
\end{align}

\newpage
\hrule
\vspace*{.2cm}
\enshu
微分方程式
\begin{align*}
  \dfrac{dy}{dx} + 2y = 2\sin x, 
\end{align*}
の一般解を求めよ.また,初期条件が$y(0)=1$のときの解を求めよ.
\vspace*{.2cm}
\hrule
\vspace*{.2cm}

\setcounter{section}{2}
\setcounter{equation}{0}
\noindent
両辺に,$e^{2x}$をかけて,
\begin{align}
 & e^{2x}\dfrac{dy}{dx} + 2e^{2x}y = 2e^{2x}\sin x, \notag \\
 & \rightarrow \dfrac{d}{dx}\left(e^{2x}y\right) = 2e^{2x}\sin x. 
\end{align}
両辺を$x$で積分すると,
\begin{align}
 e^{2x} y = 2\int dx \, e^{2x}\sin x + C. 
\end{align}
右辺の積分について考える.まず,オイラーの公式より,$\sin$関数は
\begin{align}
  \sin x = \dfrac{e^{ix} - e^{-ix}}{2i}, 
\end{align}
なので,被積分関数を
\begin{align}
 2e^{2x}\sin x = -i\left(e^{(2+i)x} - e^{(2-i)x}\right). 
\end{align}
のように表した上で,積分を実行すると,
\begin{align}
 2\int dx\,e^{2x}\sin x 
 & = -i \left(\int dx\, e^{(2+i)x} - \int dx\, e^{(2-i)x}\right) \notag \\
 & = -i \left(\dfrac{1}{2+i}e^{(2+i)x} - \dfrac{1}{2-i}e^{(2-i)x}\right) \notag \\
 & = \dfrac{2}{5}\left(2\sin x - \cos x\right)e^{2x},
\end{align}
となる.従って,
\begin{align}
  &e^{2x}y = \dfrac{2}{5}\left(2\sin x - \cos x\right)e^{2x} + C, \notag \\
  &\rightarrow y = \dfrac{2}{5}\left(2\sin x -\cos x\right) + Ce^{-2x}.
\end{align}
また,$y(0)=1$のとき,$C=\dfrac{7}{5}$なので,
\begin{align}
  y = \dfrac{2}{5}\left(2\sin x - \cos x\right)  + \dfrac{7}{5}e^{-2x}.
\end{align}

\newpage
\hrule
\vspace*{.2cm}
\enshu
化学反応
\begin{align*}
 \ce{A ->[k_1] B ->[k_2] C}  
\end{align*}
を考える.ただし,$k_1,~k_2$は各過程における速度定数である.
A, B, Cの時刻$t$での濃度をそれぞれ$\ce{[A]}_t,~\ce{[B]}_t,~\ce{[C]}_t$
とすると,反応速度式(微分方程式)は
\begin{align*}
  &\dfrac{d[\mathrm{A}]_t}{dt} = -k_1 [\mathrm{A}]_t, \\
  &\dfrac{d[\mathrm{B}]_t}{dt} =  k_1 [\mathrm{A}]_t - k_2 [\mathrm{B}]_t, \\
  &\dfrac{d[\mathrm{C}]_t}{dt} =  k_2 [\mathrm{B}]_t, 
\end{align*}
で表される.
初期濃度を$[\mathrm{A}]_0,~[\mathrm{B}]_0,~[\mathrm{C}]_0$として上記の微分方程式を解いて,
A, B, Cの濃度の時間変化を求めよ.
\vspace*{.2cm}
\hrule
\vspace*{.2cm}

\setcounter{section}{3}
\setcounter{equation}{0}
\noindent
Aに関する反応速度式は直ちに解くことが出来て,
\begin{align}
 [\mathrm{A}]_t = [\mathrm{A}]_{0}e^{-k_1 t}. 
\end{align}
これをBに関する反応速度式に代入すると,
\begin{align}
 &\dfrac{d[\mathrm{B}]_{t}}{dt} = k_1 [\mathrm{A}]_0 e^{-k_1 t} - k_2 [\mathrm{B}]_t, \notag \\
 &\rightarrow \dfrac{d[\mathrm{B}]_t}{dt} + k_2 [\mathrm{B}]_t = k_1 [\mathrm{A}]_0 e^{-k_1 t} 
\end{align}
両辺に$e^{k_2 t}$をかけると,
\begin{align}
  \dfrac{d}{dt}\left(e^{k_2 t}[\mathrm{B}]_t\right) = k_1 [\mathrm{A}]_0 e^{\left(k_2 - k_1\right)t}, 
\end{align}
となるので,両辺を$t$で積分すると,
\begin{align}
 & e^{k_2 t}[\mathrm{B}]_t = \dfrac{k_1}{k_2 - k_1}[\mathrm{A}]_0 e^{(k_2 - k_1)t} + C, \notag \\
 &\rightarrow [\mathrm{B}]_t = \dfrac{k_1}{k_2 - k_1}[\mathrm{A}]_0 e^{- k_1 t} + Ce^{-k_2 t}. 
\end{align}
Bの初期濃度は$[\mathrm{B}]_0$なので,任意定数$C$は
\begin{align}
  &[\mathrm{B}]_0 = \dfrac{k_1}{k_2 - k_1}[\mathrm{A}]_0 + C, \notag \\
  &\rightarrow C = [\mathrm{B}]_0 - \dfrac{k_1}{k_2 - k_1}[\mathrm{A}]_0,
\end{align}
となる.従って,
\begin{align}
 [\mathrm{B}]_t = \dfrac{k_1}{k_2 - k_1}[\mathrm{A}]_0 \left(e^{-k_1 t}-e^{-k_2 t}\right) + [\mathrm{B}]_0 e^{-k_2 t}. 
\end{align}
$[\mathrm{C}]_t$については,
\begin{align}
 \dfrac{d}{dt}\left([\mathrm{A}]_t + [\mathrm{B}]_t + [\mathrm{C}]_t\right) = 0 
\end{align}
つまり,
\begin{align}
 [\mathrm{A}]_t + [\mathrm{B}]_t + [\mathrm{C}]_t =[\mathrm{A}]_0 + [\mathrm{B}]_0 + [\mathrm{C}]_0,
\end{align}
であることから,
\begin{align}
 [\mathrm{C}]_t = [\mathrm{A}]_0 + [\mathrm{B}]_0 + [\mathrm{C}]_0 - [\mathrm{A}]_t - [\mathrm{B}]_t
\end{align}
と求めることができる.

\newpage
\vspace*{.2cm}
\hrule
\enshu
バネにつながれた質量$m$のおもりの運動$x(t)$を考える.
おもりの平衡位置を$x=0$,バネ定数を$k$とすると,このおもりの運動は次の微分方程式に
従うとする.
\begin{align*}
  m\dfrac{d^2x}{dt^2} = -k x. 
\end{align*}
いま,このおもりとバネを丸ごと粘性のある流体の中に入れる.
すると,おもりが動く方向とは逆向きに力(粘性力)が働く.
おもりの動く速さがあまり大きくないときは,その力はおもりの速度に比例する.ここでは比例定数を$\Gamma ~(>0)$と表すことにする.
\begin{enumerate}[(1)]
  \item 流体の中でのおもりの運動方程式を立てよ.
  \item $\omega=\sqrt{k/m}$, $\gamma = \Gamma/(2m)$とおいて,次のそれぞれの場合での(1)で立てた微分方程式の解を求めよ.
	ただし,$x(0)=1$, $dx/dt|_{t=0} = 0$とする.\\
	(i) $\gamma < \omega$, (ii) $\gamma = \omega$, (iii) $\gamma > \omega$
  \item (i), (ii), (iii)における解$x(t)$を時間$t$に対してプロットしてみて,その形について考察せよ.
	プロットは手書きで大まかに書いても良いし,ExcelやGnuplotを使っても良い.
\end{enumerate}
\vspace*{.2cm}
\hrule


