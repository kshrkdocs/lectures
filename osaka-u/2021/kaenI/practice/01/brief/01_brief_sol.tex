\documentclass[11pt,a4]{jsarticle}
\pagestyle{empty}
\setcounter{secnumdepth}{3}
\setcounter{tocdepth}{2}
\bibliographystyle{h-physrev3}
\usepackage[T1]{fontenc}
\usepackage[active]{srcltx}
\usepackage[version=3]{mhchem}

\makeatletter
%\usepackage{tgtermes}
%\usepackage[T1]{fontenc}
\usepackage[top=10truemm,bottom=10truemm,left=20truemm,right=20truemm]{geometry}
\usepackage{fancyhdr, lastpage}
\usepackage{amsmath}
%\usepackage[lite,subscriptcorrection,slantedGreek,nofontinfo]{mtpro2}
\usepackage{bm,braket,ascmac,enumerate,multirow}
\usepackage{amssymb,wrapfig,afterpage,booktabs,url}
\usepackage{listings}
\numberwithin{equation}{section}
%
\usepackage{graphicx}
\usepackage[dvips]{color}
\usepackage{makeidx}
\usepackage{fancyvrb}
\usepackage{cprotect}
\usepackage[dvipdfmx]{hyperref}
\usepackage{pxjahyper}
%
\newcommand{\vred}{\color{red}}
\newcommand{\vblue}{\color{blue}}
\newcommand{\vgreen}{\color{green}}
\newcommand{\at}{@}
%
\renewcommand{\baselinestretch}{1.1}
\renewcommand{\figurename}{Fig.}
\renewcommand{\tablename}{Table }

\makeindex 
%\renewcommand{\contentsname}{\Large \centerline{目 次}}
%\renewcommand{\figurename}{Fig.}
%\renewcommand{\tablename}{Table }
%\renewcommand{\refname}{}

\makeatother

\newcommand{ \Sec }[1]{Sec.~\ref{sec:#1}}
\newcommand{ \Appendix }[1]{Appendix \ref{sec:#1}}

\newcommand{ \Eq   }[1]{Eq.~(\ref{#1})}
\newcommand{ \Eqs  }[2]{Eqs.~(\ref{#1}) and (\ref{#2})}
\newcommand{ \Equation }[1]{Equation (\ref{#1})}

\newcommand{ \Table }[1]{Table \ref{tab:#1}}

%\newcommand{ \Ref  }[1]{Ref.~\onlinecite{#1}}
\newcommand{ \Refs }[2]{Refs.~\onlinecite{#1} and \onlinecite{#2}}

\newcommand{ \Fig     }[1]{Fig.~\ref{fig:#1}}
\newcommand{ \Figs    }[2]{Figs.~\ref{fig:#1} and \ref{fig:#2}}
\newcommand{ \Figure  }[1]{Figure \ref{fig:#1}}
%.........................................................

%.........................................................
\newtheorem{reidai}{例題}
\newtheorem{enshu}{演習問題}
%
\begin{document}
\section*{第1回 演習問題(2021/04/15) 略解とヒント}
\begin{flushright}
  最終更新日 : 2021/07/04 
\end{flushright}
%
\hrule
\vspace*{.2cm}
\enshu
放射性物質の量は時間がたつと減少していくことが知られている.
時刻$t$での放射性粒子(原子)の数を$N(t)$と表すと,
$N(t)$の時間変化率はその時刻での粒子数$N(t)$に比例することが実験的に確かめられている.
\begin{enumerate}[(1)]
  \item 比例定数を$\lambda$ ($\lambda>0$,崩壊定数と呼ばれる) として,$N(t)$に関する微分方程式を立てよ.
  \item (1)で導出した微分方程式を解け.ただし,時刻$t=0$での粒子数を$N_0$とする.
  \item 粒子数が半分になるのに要する時間$\tau$を求めよ.
\end{enumerate}
\hrule
\vspace*{.2cm}

\setcounter{section}{1}
\noindent

この問題については,ほぼ全員正解でしたので,略解のみを示します.
\begin{enumerate}[(1)]
  \item $\dfrac{dN}{dt} = -\lambda N$
  \item $N(t) = N_{0}e^{-\lambda t}$
  \item $\tau = \dfrac{1}{\lambda} \log 2 $ 
\end{enumerate}
%
\newpage
\hrule
\vspace*{.2cm}
\enshu
微分方程式
\begin{align*}
  \dfrac{dy}{dx} + 2y = 2\sin x, 
\end{align*}
の一般解を求めよ.また,初期条件が$y(0)=1$のときの解を求めよ.
\vspace*{.2cm}
\hrule
\vspace*{.2cm}

ほぼ全員正解でしたので,略解のみを示します.
\begin{align*}
  y = \dfrac{2}{5}\left(2\sin x - \cos x\right)  + \dfrac{7}{5}e^{-2x}.
\end{align*}

\newpage
\hrule
\vspace*{.2cm}
\enshu
化学反応
\begin{align*}
 \ce{A ->[k_1] B ->[k_2] C}  
\end{align*}
を考える.ただし,$k_1,~k_2$は各過程における速度定数である.
A, B, Cの時刻$t$での濃度をそれぞれ$\ce{[A]}_t,~\ce{[B]}_t,~\ce{[C]}_t$
とすると,反応速度式(微分方程式)は
\begin{align*}
  &\dfrac{d[\mathrm{A}]_t}{dt} = -k_1 [\mathrm{A}]_t, \\
  &\dfrac{d[\mathrm{B}]_t}{dt} =  k_1 [\mathrm{A}]_t - k_2 [\mathrm{B}]_t, \\
  &\dfrac{d[\mathrm{C}]_t}{dt} =  k_2 [\mathrm{B}]_t, 
\end{align*}
で表される.
初期濃度を$[\mathrm{A}]_0,~[\mathrm{B}]_0,~[\mathrm{C}]_0$として上記の微分方程式を解いて,
A, B, Cの濃度の時間変化を求めよ.
\vspace*{.2cm}
\hrule
\vspace*{.2cm}

これもほぼ全員出来ていました.

A, B, Cの時間変化は次式のように与えられます.
\begin{align*}
 &[\mathrm{A}]_t = [\mathrm{A}]_{0}e^{-k_1 t} \\ 
 &[\mathrm{B}]_t = \dfrac{k_1}{k_2 - k_1}[\mathrm{A}]_0 \left(e^{-k_1 t}-e^{-k_2 t}\right) + [\mathrm{B}]_0 e^{-k_2 t} \\
 & \left[\mathrm{C}\right]_{t}  =\left[\mathrm{A}\right]_{0}\left(1-\dfrac{k_{2}}{k_{2}-k_{1}}e^{-k_{1}t}+\dfrac{k_{1}}{k_{2}-k_{1}}e^{-k_{2}t}\right)+\left[\mathrm{B}\right]_{0}\left(1-e^{-k_{2}t}\right)+\left[\mathrm{C}\right]_{0}
\end{align*}
$[\mathrm{C}]_t$については,
\begin{align*}
 \dfrac{d}{dt}\left([\mathrm{A}]_t + [\mathrm{B}]_t + [\mathrm{C}]_t\right) = 0 
\end{align*}
つまり,
\begin{align*}
 [\mathrm{A}]_t + [\mathrm{B}]_t + [\mathrm{C}]_t =[\mathrm{A}]_0 + [\mathrm{B}]_0 + [\mathrm{C}]_0,
\end{align*}
が成り立つので,
\begin{align*}
 [\mathrm{C}]_t = [\mathrm{A}]_0 + [\mathrm{B}]_0 + [\mathrm{C}]_0 - [\mathrm{A}]_t - [\mathrm{B}]_t
\end{align*}
という関係式から求めることができます.

\newpage
\vspace*{.2cm}
\hrule
\enshu
バネにつながれた質量$m$のおもりの運動$x(t)$を考える.
おもりの平衡位置を$x=0$,バネ定数を$k$とすると,このおもりの運動は次の微分方程式に
従うとする.
\begin{align*}
  m\dfrac{d^2x}{dt^2} = -k x. 
\end{align*}
いま,このおもりとバネを丸ごと粘性のある流体の中に入れる.
すると,おもりが動く方向とは逆向きに力(粘性力)が働く.
おもりの動く速さがあまり大きくないときは,その力はおもりの速度に比例する.ここでは比例定数を$\Gamma ~(>0)$と表すことにする.
\begin{enumerate}[(1)]
  \item 流体の中でのおもりの運動方程式を立てよ.
  \item $\omega=\sqrt{k/m}$, $\gamma = \Gamma/(2m)$とおいて,次のそれぞれの場合での(1)で立てた微分方程式の解を求めよ.
	ただし,$x(0)=1$, $dx/dt|_{t=0} = 0$とする.\\
	(i) $\gamma < \omega$, (ii) $\gamma = \omega$, (iii) $\gamma > \omega$
  \item (i), (ii), (iii)における解$x(t)$を時間$t$に対してプロットしてみて,その形について考察せよ.
	プロットは手書きで大まかに書いても良いし,ExcelやGnuplotを使っても良い.
\end{enumerate}
\vspace*{.2cm}
\hrule
%
\begin{enumerate}[(1)]
\item $m\dfrac{d^2}{dt^2}x(t) = -kx(t) -\Gamma \dfrac{d}{dt}x(t)$
\item (i) $\gamma < \omega$のとき,
%
\begin{align*}
 x\left(t\right) = e^{-\gamma t}\left(\cos \tilde{\omega}t + \dfrac{\gamma}{\tilde{\omega}}\sin \tilde{\omega}t\right) 
\end{align*}
ただし,
\begin{align*}
  \tilde{\omega} = \sqrt{\omega^2 - \gamma^2}
\end{align*}
%
(ii) $\gamma = \omega$のとき,
\begin{align*}
x\left(t\right) & =e^{-\gamma t}+\gamma te^{-\gamma t}
\end{align*}
%
(iii) $\gamma > \omega$のとき,
\begin{align*}
x\left(t\right) & =-\dfrac{\gamma_{2}}{\gamma_{1}-\gamma_{2}}e^{-\gamma_{1}t}+\dfrac{\gamma_{1}}{\gamma_{1}-\gamma_{2}}e^{-\gamma_{2}t}
\end{align*}
ただし,
\begin{align*}
  \lambda_1 = \gamma + \sqrt{\gamma^2 - \omega^2}, ~\lambda_2 = \gamma - \sqrt{\gamma^2 - \omega^2} 
\end{align*}
%
\item
%
(2)で解の形を具体的に求めることが出来たわけですから,$\gamma$や$\omega$に具体的な値を入れてみて,Excelでプロットしてみましょう.
どの条件のときに,減衰は最も速くなるのでしょうか?
\end{enumerate}

\end{document}
