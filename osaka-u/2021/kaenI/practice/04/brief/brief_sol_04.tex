\documentclass[11pt,a4]{jsarticle}
\pagestyle{empty}
\setcounter{secnumdepth}{3}
\setcounter{tocdepth}{2}
\bibliographystyle{h-physrev3}
\usepackage[T1]{fontenc}
\usepackage[active]{srcltx}
\usepackage[version=3]{mhchem}

\makeatletter
%\usepackage{tgtermes}
%\usepackage[T1]{fontenc}
\usepackage[top=10truemm,bottom=10truemm,left=20truemm,right=20truemm]{geometry}
\usepackage{fancyhdr, lastpage}
\usepackage{amsmath}
%\usepackage[lite,subscriptcorrection,slantedGreek,nofontinfo]{mtpro2}
\usepackage{bm,braket,ascmac,enumerate,multirow}
\usepackage{amssymb,wrapfig,afterpage,booktabs,url}
\usepackage{listings}
\numberwithin{equation}{section}
%
\usepackage{graphicx}
\usepackage[dvips]{color}
\usepackage{makeidx}
\usepackage{fancyvrb}
\usepackage{cprotect}
\usepackage[dvipdfmx]{hyperref}
\usepackage{pxjahyper}
%
\newcommand{\vred}{\color{red}}
\newcommand{\vblue}{\color{blue}}
\newcommand{\vgreen}{\color{green}}
\newcommand{\at}{@}
%
\renewcommand{\baselinestretch}{1.1}
\renewcommand{\figurename}{Fig.}
\renewcommand{\tablename}{Table }

\makeindex 
%\renewcommand{\contentsname}{\Large \centerline{目 次}}
%\renewcommand{\figurename}{Fig.}
%\renewcommand{\tablename}{Table }
%\renewcommand{\refname}{}

\makeatother

\newcommand{ \Sec }[1]{Sec.~\ref{sec:#1}}
\newcommand{ \Appendix }[1]{Appendix \ref{sec:#1}}

\newcommand{ \Eq   }[1]{Eq.~(\ref{#1})}
\newcommand{ \Eqs  }[2]{Eqs.~(\ref{#1}) and (\ref{#2})}
\newcommand{ \Equation }[1]{Equation (\ref{#1})}

\newcommand{ \Table }[1]{Table \ref{tab:#1}}

%\newcommand{ \Ref  }[1]{Ref.~\onlinecite{#1}}
\newcommand{ \Refs }[2]{Refs.~\onlinecite{#1} and \onlinecite{#2}}

\newcommand{ \Fig     }[1]{Fig.~\ref{fig:#1}}
\newcommand{ \Figs    }[2]{Figs.~\ref{fig:#1} and \ref{fig:#2}}
\newcommand{ \Figure  }[1]{Figure \ref{fig:#1}}
%.........................................................

%.........................................................
\newtheorem{reidai}{例題}
\newtheorem{enshu}{演習問題}
%
\begin{document}
\section*{第4回 演習問題(2021/05/27) 略解とヒント}
\begin{flushright}
 最終更新日 : 2021/07/04 
\end{flushright}
\hrule
\vspace*{.2cm}
\enshu
%
次の関数$f(x)$について,フーリエ変換$F(k)$を求めよ.
\begin{enumerate}[(1)]
  \item 
	\begin{align*}
	  f(x) = e^{-a\left|x\right|}\quad (a>0)
	\end{align*}	
  \item 
	\begin{align*}
	  f(x) = 
	  \begin{cases}
	    \dfrac{1}{\epsilon} & \left|x\right| \leq \dfrac{\epsilon}{2} \\[.2cm]
            0                   & \left|x\right| >  \dfrac{\epsilon}{2}
	  \end{cases}
	\end{align*}
	ただし,$\epsilon > 0$とする.
  \item 
	\begin{align*}
	  f(x) = \cos k_0 x
	\end{align*}
  \item	
	\begin{align*}
	  f(x) = e^{-a\left|x\right|}\cos k_0 x \quad (a>0) 
	\end{align*}
  \item	
	\begin{align*}
	  f(x) = x^2 e^{-ax^2} \quad (a>0)
	\end{align*}
	ヒント:$x^2 e^{-ax^2}$の微分を考えてみよ.
\end{enumerate}
%
\hrule
\vspace*{.2cm}
%
\begin{enumerate}[(1)]
  \item 
    \begin{align*}
      F\left(k\right) = \dfrac{2a}{k^{2} + a^{2}} 
    \end{align*}
  \item
    \begin{align*}
      F\left(k\right) = \dfrac{\sin\left(\epsilon k / 2\right)}{\left(\epsilon k / 2\right)} 
    \end{align*}
  \item
    \begin{align*}
      F\left(k\right) = \pi \left(\delta\left(k-k_{0}\right)+\delta\left(k+k_{0}\right)\right)
    \end{align*}
  \item
    \begin{align*}
      F\left(k\right) = \dfrac{a}{\left(k-k_0\right)^{2}+a^{2}} + \dfrac{a}{\left(k+k_{0}\right)^{2} + a^{2}} 
    \end{align*}
  \item
    \begin{align*}
      F\left(k\right) = \sqrt{\dfrac{\pi}{a}}e^{-k^{2}/4a} \dfrac{2a - k^{2}}{4a^2} 
    \end{align*} 
\end{enumerate}

%
\newpage
%
\hrule
\enshu
以下の問いに答えよ.
\begin{enumerate}[(1)]
  \item フーリエ変換可能な偶関数$h\left(x\right)$, $c\left(x\right)$と関数$\rho\left(x\right)~(\neq 0)$からなる方程式を考える.
	ただし,$h\left(x\right)$が未知関数で,$c(x),~\rho\left(x\right)$が既知関数とする.
	いま,次式で定義される関数
	\begin{align*}
	  X\left(x,x^{\prime}\right) &= \rho\left(x\right)\delta\left(x-x^\prime\right) + \rho\left(x\right)h\left(x-x^\prime\right)\rho\left(x^{\prime}\right), \\
	  Y\left(x,x^{\prime}\right) &= \dfrac{1}{\rho\left(x\right)}\delta\left(x-x^\prime\right)-c\left(x-x^\prime\right),
	\end{align*}
	が次の関係式を満たすとする.
	\begin{align*}
	  \int_{-\infty}^{\infty}dx^{\prime\prime}\,X(x,x^{\prime\prime})Y(x^{\prime\prime},x^{\prime}) = \delta\left(x-x^{\prime}\right).
	\end{align*}
	このとき,次式が成り立つことを示せ.
	\begin{align*}
	  h\left(x-x^{\prime}\right) = c\left(x-x^{\prime}\right) + \int_{-\infty}^{\infty}dx^{\prime\prime}\,c(x-x^{\prime\prime})\rho(x^{\prime\prime})h\left(x^{\prime\prime}-x^{\prime}\right).
	\end{align*}
  \item $\rho(x)$が定数($\rho(x) = \rho$)のとき,
	(1)で導いた式のフーリエ変換を求め,それを$H(k)~(\equiv \mathcal{F}\left[h(x)\right])$について解け.
	ただし,$C(k) = \mathcal{F}\left[c(x)\right]$を用いて良い.
\end{enumerate}
%
\hrule
%\vspace*{.2cm}
\noindent
%
\begin{enumerate}[(1)]
  \item
畳み込みで表された条件式に,$X\left(x,x^{\prime}\right)$, $Y\left(x,x^{\prime}\right)$の表式を代入すると,
左辺は
\begin{align*}
 &\int_{-\infty}^{\infty} dx^{\prime\prime}\,\left\{\rho(x)\delta(x-x^{\prime\prime})+\rho(x)h(x-x^{\prime\prime})\rho(x^{\prime\prime})\right\}\left\{\dfrac{1}{\rho(x^{\prime\prime})}\delta(x^{\prime\prime}-x^{\prime}) - c\left(x^{\prime\prime}-x^{\prime}\right)\right\} \notag \\
 & = \int_{-\infty}^{\infty}dx^{\prime\prime} \dfrac{\rho(x)}{\rho\left(x^{\prime\prime}\right)}\delta(x-x^{\prime\prime})\delta(x^{\prime\prime}-x^{\prime}) \,\cdots \textcircled{1}\notag \\ 
 & \quad - \int_{-\infty}^{\infty}dx^{\prime\prime}\rho(x)\delta(x-x^{\prime\prime})c(x^{\prime\prime}-x^{\prime}) 
 \cdots \textcircled{2}\notag \\
 & \quad +\int_{-\infty}^{\infty}dx^{\prime\prime}\,\rho(x)h(x-x^{\prime\prime})\delta(x^{\prime\prime}-x^{\prime})\,\cdots \textcircled{3} \notag \\
 & \quad -\int_{-\infty}^{\infty}dx^{\prime\prime}\,\rho(x)h(x-x^{\prime\prime})\rho(x^{\prime\prime})c(x^{\prime\prime}-x^{\prime}) \,\cdots \textcircled{4}
\end{align*}
となる.それぞれの項を計算すると,
\begin{align*}
 \textcircled{1} &= \delta(x-x^{\prime}) \\
 \textcircled{2} &= -\rho(x)c(x-x^{\prime}) \\
 \textcircled{3} &= \rho(x)h(x-x^{\prime}) \\
 \textcircled{4} &= -\rho(x)\int_{-\infty}^{\infty}dx^{\prime\prime}\,h(x-x^{\prime\prime})\rho(x^{\prime\prime})c(x^{\prime\prime}-x^{\prime})
\end{align*}
従って,条件式は
\begin{align*}
 h(x-x^{\prime}) &= c(x-x^{\prime}) + \int_{-\infty}^{\infty}dx^{\prime\prime}\,
                   h(x-x^{\prime\prime})\rho(x^{\prime\prime})c(x^{\prime\prime}-x^{\prime}) \\
                 &= c(x-x^{\prime}) + \int_{-\infty}^{\infty}dx^{\prime\prime}\, c(x^{\prime}-x^{\prime\prime})\rho(x^{\prime\prime})h(x^{\prime\prime}-x)
\end{align*}
と表せる.$x$, $x^{\prime}$を入れ替えた式は,
\begin{align*}
  h(x^\prime -x) = c(x^\prime - x) + \int_{-\infty}^{\infty}dx^{\prime\prime}\,c(x-x^{\prime\prime})\rho(x^{\prime\prime})h(x^{\prime\prime}-x^{\prime}) 
\end{align*}
であり,$h(x)$と$c(x)$は偶関数なので,最終的に
\begin{align*}
  h(x-x^{\prime}) = c(x-x^{\prime}) + \int_{-\infty}^{\infty}dx^{\prime\prime}\,c(x-x^{\prime\prime})\rho(x^{\prime\prime})h(x^{\prime\prime}-x^{\prime}) 
\end{align*}
を得る.
%
  \item
%
    \begin{align*}
      H\left(k\right) = \dfrac{C\left(k\right)}{1-\rho C\left(k\right)} 
    \end{align*}
    この問題は正答率50\%くらいでした.
    畳み込み積分のフーリエ変換をただ$C(k)H(k)$としている人がいましたが,正しく式変形すれば,これ以外の因子が現れるはずです.
\end{enumerate}
\end{document}
