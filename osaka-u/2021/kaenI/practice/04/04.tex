\section*{第4回 演習問題(2021/05/27)}
%
\enshu
%
次の関数$f(x)$について,フーリエ変換$F(k)$を求めよ.
\begin{enumerate}[(1)]
  \item 
	\begin{align*}
	  f(x) = e^{-a\left|x\right|}\quad (a>0)
	\end{align*}	
  \item 
	\begin{align*}
	  f(x) = 
	  \begin{cases}
	    \dfrac{1}{\epsilon} & \left|x\right| \leq \dfrac{\epsilon}{2} \\[.2cm]
            0                   & \left|x\right| >  \dfrac{\epsilon}{2}
	  \end{cases}
	\end{align*}
	ただし,$\epsilon > 0$とする.
  \item 
	\begin{align*}
	  f(x) = \cos k_0 x
	\end{align*}
  \item	
	\begin{align*}
	  f(x) = e^{-a\left|x\right|}\cos k_0 x \quad (a>0) 
	\end{align*}
  \item	
	\begin{align*}
	  f(x) = x^2 e^{-ax^2} \quad (a>0)
	\end{align*}
	ヒント:$x^2 e^{-ax^2}$の微分を考えてみよ.
\end{enumerate}
%
\enshu
以下の問いに答えよ.
\begin{enumerate}[(1)]
  \item フーリエ変換可能な偶関数$h\left(x\right)$, $c\left(x\right)$と関数$\rho\left(x\right)~(\neq 0)$からなる方程式を考える.
	ただし,$h\left(x\right)$が未知関数で,$c(x),~\rho\left(x\right)$が既知関数とする.
	いま,次式で定義される関数
	\begin{align*}
	  X\left(x,x^{\prime}\right) &= \rho\left(x\right)\delta\left(x-x^\prime\right) + \rho\left(x\right)h\left(x-x^\prime\right)\textcolor{blue}{\rho\left(x^{\prime}\right)}, \\
	  Y\left(x,x^{\prime}\right) &= \dfrac{1}{\rho\left(x\right)}\delta\left(x-x^\prime\right)-c\left(x-x^\prime\right),
	\end{align*}
	が次の関係式を満たすとする.
	\textcolor{red}{
	\begin{align*}
	  \int_{-\infty}^{\infty}dx^{\prime\prime}\,X(x,x^{\prime\prime})Y(x^{\prime\prime},x^{\prime}) = \delta\left(x-x^{\prime}\right).
	\end{align*}
	}
	このとき,次式が成り立つことを示せ.
	\begin{align*}
	  h\left(x-x^{\prime}\right) = c\left(x-x^{\prime}\right) + \int_{-\infty}^{\infty}dx^{\prime\prime}\,c(x-x^{\prime\prime})\rho(x^{\prime\prime})h\left(x^{\prime\prime}-x^{\prime}\right).
	\end{align*}
  \item $\rho(x)$が定数($\rho(x) = \rho$)のとき,
	(1)で導いた式のフーリエ変換を求め,それを$H(k)~(\equiv \mathcal{F}\left[h(x)\right])$について解け.
	ただし,$C(k) = \mathcal{F}\left[c(x)\right]$を用いて良い.
\end{enumerate}

\noindent
\textcolor{red}{
\fbox{訂正}
\quad 演習問題2の条件式(赤字)を訂正しました.講義時のプリントでは積分変数が$x$になっていたのですが,正しくは,上で示しているように$x^{\prime\prime}$です.申し訳ありませんでした.
}
\\
\textcolor{blue}{
さらに,$X(x,x^{\prime})$の式に誤りがあったので修正しました.何度も申し訳ありません.
こちらの不手際が何度もありましたし,この問題(1)は例題とし,次のページに解答を掲載します.
}
%
\newpage
%
\noindent
\fbox{演習問題2 (1)の解答}

\noindent
畳み込みで表された条件式に,$X\left(x,x^{\prime}\right)$, $Y\left(x,x^{\prime}\right)$の表式を代入すると,
左辺は
\begin{align*}
 &\int_{-\infty}^{\infty} dx^{\prime\prime}\,\left\{\rho(x)\delta(x-x^{\prime\prime})+\rho(x)h(x-x^{\prime\prime})\rho(x^{\prime\prime})\right\}\left\{\dfrac{1}{\rho(x^{\prime\prime})}\delta(x^{\prime\prime}-x^{\prime}) - c\left(x^{\prime\prime}-x^{\prime}\right)\right\} \notag \\
 & = \int_{-\infty}^{\infty}dx^{\prime\prime} \dfrac{\rho(x)}{\rho\left(x^{\prime\prime}\right)}\delta(x-x^{\prime\prime})\delta(x^{\prime\prime}-x^{\prime}) \,\cdots \textcircled{1}\notag \\ 
 & \quad - \int_{-\infty}^{\infty}dx^{\prime\prime}\rho(x)\delta(x-x^{\prime\prime})c(x^{\prime\prime}-x^{\prime}) 
 \cdots \textcircled{2}\notag \\
 & \quad +\int_{-\infty}^{\infty}dx^{\prime\prime}\,\rho(x)h(x-x^{\prime\prime})\delta(x^{\prime\prime}-x^{\prime})\,\cdots \textcircled{3} \notag \\
 & \quad -\int_{-\infty}^{\infty}dx^{\prime\prime}\,\rho(x)h(x-x^{\prime\prime})\rho(x^{\prime\prime})c(x^{\prime\prime}-x^{\prime}) \,\cdots \textcircled{4}
\end{align*}
となる.それぞれの項を計算すると,
\begin{align*}
 \textcircled{1} &= \delta(x-x^{\prime}) \\
 \textcircled{2} &= -\rho(x)c(x-x^{\prime}) \\
 \textcircled{3} &= \rho(x)h(x-x^{\prime}) \\
 \textcircled{4} &= -\rho(x)\int_{-\infty}^{\infty}dx^{\prime\prime}\,h(x-x^{\prime\prime})\rho(x^{\prime\prime})c(x^{\prime\prime}-x^{\prime})
\end{align*}
従って,条件式は
\begin{align*}
 h(x-x^{\prime}) &= c(x-x^{\prime}) + \int_{-\infty}^{\infty}dx^{\prime\prime}\,
                   h(x-x^{\prime\prime})\rho(x^{\prime\prime})c(x^{\prime\prime}-x^{\prime}) \\
                 &= c(x-x^{\prime}) + \int_{-\infty}^{\infty}dx^{\prime\prime}\, c(x^{\prime}-x^{\prime\prime})\rho(x^{\prime\prime})h(x^{\prime\prime}-x)
\end{align*}
と表せる.$x$, $x^{\prime}$を入れ替えた式は,
\begin{align*}
  h(x^\prime -x) = c(x^\prime - x) + \int_{-\infty}^{\infty}dx^{\prime\prime}\,c(x-x^{\prime\prime})\rho(x^{\prime\prime})h(x^{\prime\prime}-x^{\prime}) 
\end{align*}
であり,$h(x)$と$c(x)$は偶関数なので,最終的に
\begin{align*}
  h(x-x^{\prime}) = c(x-x^{\prime}) + \int_{-\infty}^{\infty}dx^{\prime\prime}\,c(x-x^{\prime\prime})\rho(x^{\prime\prime})h(x^{\prime\prime}-x^{\prime}) 
\end{align*}
を得る.