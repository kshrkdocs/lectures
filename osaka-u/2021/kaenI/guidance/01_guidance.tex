\documentclass[11pt,a4]{jsarticle}
\pagestyle{empty}
\setcounter{secnumdepth}{3}
\setcounter{tocdepth}{2}
\bibliographystyle{h-physrev3}
\usepackage[T1]{fontenc}
\usepackage[active]{srcltx}
\usepackage[version=3]{mhchem}

\makeatletter
%\usepackage{tgtermes}
%\usepackage[T1]{fontenc}
\usepackage[top=10truemm,bottom=10truemm,left=20truemm,right=20truemm]{geometry}
\usepackage{fancyhdr, lastpage}
\usepackage{amsmath}
%\usepackage[lite,subscriptcorrection,slantedGreek,nofontinfo]{mtpro2}
\usepackage{bm,braket,ascmac,enumerate,multirow}
\usepackage{amssymb,wrapfig,afterpage,booktabs,url}
\usepackage{listings}
\numberwithin{equation}{section}
%
\usepackage{graphicx}
\usepackage[dvips]{color}
\usepackage{makeidx}
\usepackage{fancyvrb}
\usepackage{cprotect}
\usepackage[dvipdfmx]{hyperref}
\usepackage{pxjahyper}
%
\newcommand{\vred}{\color{red}}
\newcommand{\vblue}{\color{blue}}
\newcommand{\vgreen}{\color{green}}
\newcommand{\at}{@}
%
\renewcommand{\baselinestretch}{1.1}
\renewcommand{\figurename}{Fig.}
\renewcommand{\tablename}{Table }

\makeindex 
%\renewcommand{\contentsname}{\Large \centerline{目 次}}
%\renewcommand{\figurename}{Fig.}
%\renewcommand{\tablename}{Table }
%\renewcommand{\refname}{}

\makeatother

\newcommand{ \Sec }[1]{Sec.~\ref{sec:#1}}
\newcommand{ \Appendix }[1]{Appendix \ref{sec:#1}}

\newcommand{ \Eq   }[1]{Eq.~(\ref{#1})}
\newcommand{ \Eqs  }[2]{Eqs.~(\ref{#1}) and (\ref{#2})}
\newcommand{ \Equation }[1]{Equation (\ref{#1})}

\newcommand{ \Table }[1]{Table \ref{tab:#1}}

%\newcommand{ \Ref  }[1]{Ref.~\onlinecite{#1}}
\newcommand{ \Refs }[2]{Refs.~\onlinecite{#1} and \onlinecite{#2}}

\newcommand{ \Fig     }[1]{Fig.~\ref{fig:#1}}
\newcommand{ \Figs    }[2]{Figs.~\ref{fig:#1} and \ref{fig:#2}}
\newcommand{ \Figure  }[1]{Figure \ref{fig:#1}}
%.........................................................

%.........................................................
\newtheorem{reidai}{例題}
\newtheorem{enshu}{演習問題}
%
\begin{document}
\section*{第1回 化学工学演習I(笠原担当分) 2021/04/15}
\begin{flushright}
  大阪大学基礎工学研究科 物質創成専攻 化学工学領域\\
  分子集合系化学工学グループ (松林研究室) \\
  助教 笠原 健人 \\
  居室: 基礎工学棟C244 \\
  Mail: \verb|kasahara@cheng.es.osaka-u.ac.jp| \\
  Tel.: 06-6850-6567 
\end{flushright}
%
\subsection*{スケジュール}
%
\begin{itemize}
  \item 4月15日 ガイダンス,常微分方程式
  \item 4月22日 常微分方程式
  \item 5月6日  岡野先生担当日
  \item 5月13日 岡野先生担当日
  \item 5月20日 偏微分方程式
  \item 5月27日 偏微分方程式
  \item 6月3日  岡野先生担当日
  \item 6月10日 岡野先生担当日
  \item 6月17日 ラプラス変換
  \item 6月24日 ラプラス変換
  \item 7月1日  岡野先生担当日
  \item 7月8日  岡野先生担当日
  \item 7月15日 ラプラス変換
  \item 7月29日 岡野先生担当日 
\end{itemize}
%
笠原は主に微分方程式関連の物理・化学数学を担当.
岡野先生は主に流体に関わる微分方程式及び
数値解法について担当.
(岡野先生と笠原の講義の内容はほぼ独立)
%
\subsection*{講義形式について}
%
各回で扱う内容を述べた後,残り時間で演習を行う.演習問題の解答はレポートとして,次の週の水曜日までに電子ファイルとして提出.
%
\subsection*{課題提出について}
%
送付先は\verb|kasahara-kadai@cheng.es.osaka-u.ac.jp|.
件名は必ず「第$x$回化工演習I(学籍番号 氏名)」にすること.本文はなくても構わない.

レポート作成は手書きとPC作成の両方を認める.
手書きの場合はスキャナー等で読み取ったものを送付すること.
PCで作成する場合は,WordかLaTeXを用いること(LaTeXの利用を強く推奨).
いずれの場合においてもPDF形式で送付すること.
%
\subsection*{Teaching Assistant (TA)}
%
八十島,沖田,竹本 (全員松林研究室学生)
%
\subsection*{内容に関する質問等について}
%
まずTAに質問すること.
ただし,講義に対するリクエストについては直接で可.
\end{document}
